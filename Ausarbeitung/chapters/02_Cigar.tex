%
% Einleitung
%

\chapter{CIGAR-Strings}

Ein Dateiformat, welches zur Speicherung von Alignments verwendet wird, ist das SAM-Format oder die komprimierte Version BAM. Dieses codiert ein Alignment in einem sogenannten Cigar-String der aus einzelnen Zeichen besteht, die jeweil eine Edit-Operation bezeichnen, also M f�r eine Substitution, I f�r eine Insertion und D f�r eine Deletion. Gleiche aufeinanderfolgende Operationen werden als Kombination von Quantit�t und Symbol geschrieben.

Beispiel 1: Sei $u =$ \texttt{actgaact}, $v =$ \texttt{actagaat} und das Alignment $A = (a\rightarrow a, c\rightarrow c, t\rightarrow t, ...)$ gegeben.
\begin{center}
	\texttt{
		\begin{tabular}{ccccccccc}
			a & c & t & - & g & a & a & c & t\\
			$|$&$|$&$|$&&$|$&$|$&$|$& &$|$\\
			a&c&t&a&g&a&a& - &t
		\end{tabular}
	}
\end{center}
Ein Alignment wird �blicherweise in drei Zeilen geschrieben, wobei in der ersten Zeile die Sequenz $u$ und in der dritten Zeile die Sequenz $v$ geschrieben wird. In der mittleren Zeile symbolisiert das Zeichen '$|$' eine Substitution, wobei �blicherweise nur ein Match markiert wird. Au�erdem wird ein $\varepsilon$ aus der Edit-Operation in diesem Fall durch das Zeichen '-' dargestellt.

Dieses Alignment wird duch den Cigar-String \texttt{3M1I3M1D1M} repr�sentiert. Das Format ben�tigt wenig Speicher f�r Alignments mit einer kleinen Edit-Distanz und deutlich mehr Speicher f�r Alignments mit einer gro�en Edit-Distanz \cite{sam}.

\section{Komplexit�t}
\clearpage

\section{Speicherverbrauch}\label{Cigar_Speicher}
\subsection{Beispiel}
Sei das Alignment $\cal{A}$
\begin{verbatim}
0    5    0    5    0    5    0    5    0    
gagc-a-t-gttgcc-tggtcctttgctaggtactgta-gaga
|| | | | |  | | ||| |||| ||| ||| ||||| ||||
gaccaagtag--g-cgtggacctt-gctcggt-ctgtaagaga
0    5    0    5    0    5    0    5    0   
\end{verbatim}
und der dazugeh�rige CIGAR-String \texttt{4M1D1M1D1M1D1M2I1M1I1M1D8M1I7M1I5M1D4M} gegeben.

Mit einer naiven bin�ren Kodierung br�uchte man demnach f�r jedes Symbol $c \in \cal{A}$ $\log_2(38) = 6$ Bit pro Symbol, also $38 \cdot 6 = 228$ Bit insgesamt.

Bei einer \textit{minimalen} bin�ren Kodierung, wie sie auch im Huffman-Alogrithmus verwendet wird, werden die L�ngen der Codew�rter anhand der relativen Wahrscheinlichkeit des Symbols im Alphabet angepasst. Somit l�sst sich eine Kodierung erm�glichen, welche im Durchschnitt weniger Bits pro Symbol beansprucht \cite{coding}.

Der Huffman-Algorithmus w�rde bei dem oben genannten Beispiel des CIGAR-Strings wie folgt ablaufen:

Gegeben sei das Alphabet $\cal{A}$ $= \{M, I, D, 1, 2, 4, 5, 7, 8\}$, sowie die relativen Wahrscheinlichkeiten $p(i)$ jeden Symbols $i \in$ $\cal{A}$ mit

\begin{center}
	\begin{tabular}{ccc}
		$i$&Anzahl&$p(i)$\\
		\hline
		1 & 13 &$\frac{13}{38}$\\
		M & 10 & $\frac{10}{38}$\\
		D & 5 & $\frac{5}{38}$\\
		I & 4 & $\frac{4}{38}$\\
		4 & 2 & $\frac{2}{38}$\\
		2 & 1 & $\frac{1}{38}$\\
		5 & 1 & $\frac{1}{38}$\\
		7 & 1 & $\frac{1}{38}$\\
		8 & 1 & $\frac{1}{38}$
	\end{tabular}
\end{center}

Ausf�hrung des Huffman-Algorithmus ergibt nach \cite[S. 54]{coding}:

\begin{center}
	\begin{tabular}{crr}
		Symbol & Kodierung & Anzahl Bits\\
		\hline
		1 & 00 & 2\\
		M & 01 & 2\\
		D & 100 & 3\\
		I & 110 & 3\\
		4 & 1010 & 4\\
		7 & 1110 & 4\\
		8 & 1111 & 4\\
		2 & 10110 & 5\\
		5 & 10111 & 5\\
		\hline
		\multicolumn{2}{l}{Gesamtanzahl:}&32
	\end{tabular}
\end{center}

und der dazugeh�rige Huffman-Baum:

\begin{center}
\begin{forest}
	for tree={grow'=south}
	[$\frac{38}{38}$
	[$\frac{15}{38}$, edge label={node[midway,right,font=\scriptsize]{1}}
	[$\frac{6}{38}$, edge label={node[midway,right,font=\scriptsize]{1}}
	[$\frac{2}{38}$, edge label={node[midway,right,font=\scriptsize]{1}}
	[8, edge label={node[midway,right,font=\scriptsize]{1}}]
	[7, edge label={node[midway,left,font=\scriptsize]{0}}] ]
	[I, edge label={node[midway,left,font=\scriptsize]{0}}] ]
	[$\frac{9}{38}$, edge label={node[midway,left,font=\scriptsize]{0}}
	[$\frac{4}{38}$, edge label={node[midway,right,font=\scriptsize]{1}}
	[$\frac{2}{38}$, edge label={node[midway,right,font=\scriptsize]{1}}
	[5, edge label={node[midway,right,font=\scriptsize]{1}}]
	[2, edge label={node[midway,left,font=\scriptsize]{0}}] ]
	[4, edge label={node[midway,left,font=\scriptsize]{0}}] ]
	[D, edge label={node[midway,left,font=\scriptsize]{0}}] ] ]
	[$\frac{23}{38}$, edge label={node[midway,left,font=\scriptsize]{0}} 
	[M, edge label={node[midway,right,font=\scriptsize]{1}}]
	[1, edge label={node[midway,left,font=\scriptsize]{0}}] ]
	]
\end{forest}
\end{center}

Der gesamte Bitverbrauch dieser Kodierung ist demnach 32 Bit.

Der durchschnittliche Bitverbrauch f�r ein Symbol betr�gt 
\[\frac{38 + 23 + 15 + 9 + 6 + 4 + 2 + 2}{38} \approx 2.61 \text{ Bit.}\]

\subsection{Testl�ufe}

Alle folgenden Grafiken wurden mit den folgenden Parametern berechnet:

Jeweils 100 zuf�llig generierte Sequenzpaare mit 200 Basen, einer Fehlerrate von 15\% und einem $\Delta$-Wert von 10.

\begin{filecontents*}{cigar_huffman.txt}
	1 45
	2 26
	3 32
	4 31
	5 39
	6 32
	7 27
	8 37
	9 32
	10 36
	11 36
	12 45
	13 37
	14 27
	15 39
	16 32
	17 38
	18 39
	19 27
	20 44
	21 32
	22 43
	23 32
	24 38
	25 38
	26 45
	27 32
	28 39
	29 38
	30 26
	31 32
	32 38
	33 37
	34 32
	35 27
	36 32
	37 37
	38 45
	39 39
	40 43
	41 26
	42 44
	43 38
	44 43
	45 32
	46 26
	47 45
	48 21
	49 50
	50 39
	51 38
	52 37
	53 25
	54 52
	55 29
	56 21
	57 49
	58 26
	59 43
	60 45
	61 44
	62 39
	63 26
	64 41
	65 32
	66 39
	67 43
	68 38
	69 37
	70 26
	71 45
	72 38
	73 32
	74 47
	75 38
	76 32
	77 32
	78 43
	79 43
	80 36
	81 21
	82 45
	83 50
	84 52
	85 45
	86 37
	87 32
	88 48
	89 57
	90 36
	91 32
	92 52
	93 46
	94 45
	95 39
	96 57
	97 52
	98 44
	99 38
	100 51
\end{filecontents*}

\begin{filecontents*}{cigar_binary.txt}
	1 115
	2 52
	3 150
	4 234
	5 294
	6 306
	7 155
	8 252
	9 246
	10 56
	11 100
	12 120
	13 258
	14 198
	15 360
	16 246
	17 135
	18 294
	19 276
	20 222
	21 115
	22 258
	23 252
	24 52
	25 90
	26 95
	27 210
	28 240
	29 105
	30 135
	31 234
	32 90
	33 198
	34 52
	35 140
	36 85
	37 160
	38 288
	39 85
	40 155
	41 110
	42 110
	43 360
	44 110
	45 135
	46 210
	47 258
	48 110
	49 160
	50 324
	51 160
	52 90
	53 150
	54 150
	55 36
	56 222
	57 270
	58 216
	59 234
	60 240
	61 135
	62 135
	63 288
	64 258
	65 135
	66 264
	67 56
	68 130
	69 294
	70 135
	71 210
	72 216
	73 228
	74 110
	75 216
	76 110
	77 110
	78 150
	79 155
	80 234
	81 198
	82 90
	83 155
	84 95
	85 264
	86 366
	87 342
	88 135
	89 372
	90 240
	91 354
	92 288
	93 324
	94 135
	95 100
	96 294
	97 160
	98 270
	99 90
	100 210
\end{filecontents*}

\begin{center}
	\begin{tikzpicture}
	\begin{axis}[
	title={Bitverbrauch der Huffman-Kodierung f�r CIGAR-Strings},			
	xlabel={Anzahl Durchl�ufe},
	ylabel={Anzahl Bits}]
	\addplot table {cigar_huffman.txt};
	\end{axis}
	\end{tikzpicture}
\end{center}

\begin{center}
	\begin{tikzpicture}
	\begin{axis}[
	title={Bitverbrauch der naiven bin�ten Kodierung f�r CIGAR-Strings},			
	xlabel={Anzahl Durchl�ufe},
	ylabel={Anzahl Bits}]
	\addplot table {cigar_binary.txt};
	\end{axis}
	\end{tikzpicture}
\end{center}

\begin{center}
	\begin{tikzpicture}
	\begin{axis}[
	title={Bitverbrauch der beiden Kodierungs-Verfahren f�r CIGAR-Strings},			
	xlabel={Anzahl Durchl�ufe},
	ylabel={Anzahl Bits}]
	\addplot table {cigar_huffman.txt};
	\addplot table {cigar_binary.txt};
	\end{axis}
	\end{tikzpicture}
\end{center}