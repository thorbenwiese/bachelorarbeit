%
% Einleitung
%

\chapter{CIGAR-Strings}

Ein Dateiformat, welches zur Speicherung von Alignments verwendet wird, ist das SAM-Format oder die komprimierte Version BAM. Dieses codiert ein Alignment in einem sogenannten Cigar-String der aus einzelnen Zeichen besteht, die jeweil eine Edit-Operation bezeichnen, also M f�r eine Substitution, I f�r eine Insertion und D f�r eine Deletion. Gleiche aufeinanderfolgende Operationen werden als Kombination von Quantit�t und Symbol geschrieben.

Beispiel 1: Sei $u =$ \texttt{actgaact}, $v =$ \texttt{actagaat} und das Alignment $A = (a\rightarrow a, c\rightarrow c, t\rightarrow t, ...)$ gegeben.
\begin{center}
	\texttt{
		\begin{tabular}{ccccccccc}
			a & c & t & - & g & a & a & c & t\\
			$|$&$|$&$|$&&$|$&$|$&$|$& &$|$\\
			a&c&t&a&g&a&a& - &t
		\end{tabular}
	}
\end{center}
Ein Alignment wird �blicherweise in drei Zeilen geschrieben, wobei in der ersten Zeile die Sequenz $u$ und in der dritten Zeile die Sequenz $v$ geschrieben wird. In der mittleren Zeile symbolisiert das Zeichen '$|$' eine Substitution, wobei �blicherweise nur ein Match markiert wird. Au�erdem wird ein $\varepsilon$ aus der Edit-Operation in diesem Fall durch das Zeichen '-' dargestellt.

Dieses Alignment wird duch den Cigar-String \texttt{3M1I3M1D1M} repr�sentiert. Das Format ben�tigt wenig Speicher f�r Alignments mit einer kleinen Edit-Distanz und deutlich mehr Speicher f�r Alignments mit einer gro�en Edit-Distanz \cite{sam}.

\section{Komplexit�t}
\clearpage

\section{Speicherverbrauch}\label{Cigar_Speicher}
Sei das Alignment $\cal{A}$
\begin{verbatim}
0    5    0    5    0    5    0    5    0    
gagc-a-t-gttgcc-tggtcctttgctaggtactgta-gaga
|| | | | |  | | ||| |||| ||| ||| ||||| ||||
gaccaagtag--g-cgtggacctt-gctcggt-ctgtaagaga
0    5    0    5    0    5    0    5    0   
\end{verbatim}
und der dazugeh�rige CIGAR-String \texttt{4M1D1M1D1M1D1M2I1M1I1M1D8M1I7M1I5M1D4M} gegeben.

Mit einer naiven bin�ren Kodierung br�uchte man demnach f�r jedes Symbol $c \in \cal{A}$ $\log_2(38) = 6$ Bit pro Symbol, also $38 \cdot 6 = 228$ Bit insgesamt.

Bei einer \textit{minimalen} bin�ren Kodierung, wie sie auch im Huffman-Alogrithmus verwendet wird, werden die L�ngen der Codew�rter anhand der relativen Wahrscheinlichkeit des Symbols im Alphabet angepasst. Somit l�sst sich eine Kodierung erm�glichen, welche im Durchschnitt weniger Bits pro Symbol beansprucht \cite{coding}.

Der Huffman-Algorithmus w�rde bei dem oben genannten Beispiel der TracePoints wie folgt ablaufen:

Gegeben sei das Alphabet $\cal{A}$ $= \{M, I, D, 1, 2, 4, 5, 7, 8\}$, sowie die relativen Wahrscheinlichkeiten $p(i)$ jeden Symbols aus $\cal{A}$ mit

\begin{tabular}{lcl}
$p(1)$&=& $\frac{13}{38}$\\
$p(M)$ &=& $\frac{10}{38}$\\
$p(D)$ &=& $\frac{5}{38}$\\
$p(I)$ &=& $\frac{4}{38}$\\
$p(4)$ &=& $\frac{2}{38}$\\
$p(2)$ &=& $\frac{1}{38}$\\
$p(5)$ &=& $\frac{1}{38}$\\
$p(7)$ &=& $\frac{1}{38}$\\
$p(8)$ &=& $\frac{1}{38}$
\end{tabular}

Ausf�hrung des Huffman-Algorithmus:

\begin{tabular}{llc}
	$\frac{13}{38}$ & $c_1 = \lambda$&\\
	$\frac{10}{38}$ & $c_2 = \lambda$&\\
	$\frac{5}{38}$ & $c_3 = \lambda$&\\
	$\frac{4}{38}$ & $c_4 = \lambda$&$\Rightarrow$\\
	$\frac{2}{38}$ & $c_5 = \lambda$&\\
	$\frac{1}{38}$ & $c_6 = \lambda$&\\
	$\frac{1}{38}$ & $c_7 = \lambda$&\\
	$\frac{1}{38}$ & $c_8 = \lambda$&\\
	$\frac{1}{38}$ & $c_9 = \lambda$&
\end{tabular}\begin{tabular}{llc}
	$\frac{13}{38}$ & $c_1 = \lambda$&\\
	$\frac{10}{38}$ & $c_2 = \lambda$&\\
	$\frac{5}{38}$ & $c_3 = \lambda$&\\
	$\frac{4}{38}$ & $c_4 = \lambda$&$\Rightarrow$\\
	$\frac{2}{38}$ & $c_5 = \lambda$&\\
	$\frac{2}{38}$ & $c_6 = 0, c_7 = 1$&\\
	$\frac{1}{38}$ & $c_8 = \lambda$&\\
	$\frac{1}{38}$ & $c_9 = \lambda$&\\
	&&
\end{tabular}\begin{tabular}{llc}
	$\frac{13}{38}$ & $c_1 = \lambda$&\\
	$\frac{10}{38}$ & $c_2 = \lambda$&\\
	$\frac{5}{38}$ & $c_3 = \lambda$&\\
	$\frac{4}{38}$ & $c_4 = \lambda$&$\Rightarrow$\\
	$\frac{2}{38}$ & $c_5 = \lambda$&\\
	$\frac{2}{38}$ & $c_6 = 0, c_7 = 1$&\\
	$\frac{2}{38}$ & $c_8 = 0, c_9 = 1$&\\
	&&\\
	&&
\end{tabular}\begin{tabular}{llc}
	$\frac{13}{38}$ & $c_1 = \lambda$&\\
	$\frac{10}{38}$ & $c_2 = \lambda$&\\
	$\frac{5}{38}$ & $c_3 = \lambda$&\\
	$\frac{4}{38}$ & $c_4 = \lambda$&$\Rightarrow$\\
	$\frac{4}{38}$ & $c_5 = 0, c_6 = 10,$&\\
	&$c_7 = 11$&\\
	$\frac{2}{38}$ & $c_8 = 0, c_9 = 1$&\\
	&&\\
	&&
\end{tabular}

\begin{tabular}{llc}
	$\frac{13}{38}$ & $c_1 = \lambda$&\\
	$\frac{10}{38}$ & $c_2 = \lambda$&\\
	$\frac{5}{38}$ & $c_3 = \lambda$&$\Rightarrow$\\
	$\frac{4}{38}$ & $c_4 = \lambda$&\\
	$\frac{4}{38}$ & $c_5 = 0, c_6 = 10, c_7 = 11$&\\
	$\frac{2}{38}$ & $c_8 = 0, c_9 = 1$&\\
	&&\\
	&&\\
	&&
\end{tabular}\begin{tabular}{llc}
	$\frac{13}{38}$ & $c_1 = \lambda$&\\
	$\frac{10}{38}$ & $c_2 = \lambda$&\\
	$\frac{6}{38}$ & $c_4 = 0, c_8 = 10, c_9 = 11$&$\Rightarrow$\\
	$\frac{5}{38}$ & $c_3 = \lambda$&\\
	$\frac{4}{38}$ & $c_5 = 0, c_6 = 10, c_7 = 11$&\\
	&&\\
	&&\\
	&&\\
	&&
\end{tabular}

\begin{tabular}{llc}
	$\frac{13}{38}$ & $c_1 = \lambda$&\\
	$\frac{10}{38}$ & $c_2 = \lambda$&\\
	$\frac{9}{38}$ & $c_3 = 0, c_5 = 10, c_6 = 110, c_7 = 111$&$\Rightarrow$\\
	$\frac{6}{38}$ & $c_4 = 0, c_8 = 10, c_9 = 11$&
\end{tabular}\begin{tabular}{llc}
	$\frac{15}{38}$ & $c_3 = 00, c_5 = 010, c_6 = 0110, c_7 = 0111,$&\\
	& $c_4 = 10, c_8 = 110, c_9 = 111$&\\
	$\frac{13}{38}$ & $c_1 = \lambda$&\\
	$\frac{10}{38}$ & $c_2 = \lambda$&$\Rightarrow$\\
	&&
\end{tabular}

\begin{tabular}{llc}
	$\frac{23}{38}$ & $c_1 = 0, c_2 = 1$&\\
	$\frac{15}{38}$ & $c_3 = 00, c_5 = 010, c_6 = 0110, c_7 = 0111,$&$\Rightarrow$\\
	& $c_4 = 10, c_8 = 110, c_9 = 111$&\\
	&&\\
	&&
\end{tabular}\begin{tabular}{ll}
$\frac{38}{38}$ & $c_1 = 00, c_2 = 01, c_3 = 100,  c_4 = 110,$\\
 &$c_5 = 1010, c_6 = 10110, c_7 = 10111,$\\
 &$c_8 = 1110, c_9 = 1111$\\
&\\
&
\end{tabular}

Somit ergibt sich die Menge $C$ der Codew�rter mit
\[C = "00", "01", "100", "110", "1010", "10110", "10111", "1110", "1111"\]
und der dazugeh�rige Huffman-Baum:

\begin{center}
\begin{forest}
	for tree={grow'=south}
	[$\frac{38}{38}$
	[$\frac{15}{38}$, edge label={node[midway,right,font=\scriptsize]{1}}
	[$\frac{6}{38}$, edge label={node[midway,right,font=\scriptsize]{1}}
	[$\frac{2}{38}$, edge label={node[midway,right,font=\scriptsize]{1}}
	[$c_9$, edge label={node[midway,right,font=\scriptsize]{1}}]
	[$c_8$, edge label={node[midway,left,font=\scriptsize]{0}}] ]
	[$c_4$, edge label={node[midway,left,font=\scriptsize]{0}}] ]
	[$\frac{9}{38}$, edge label={node[midway,left,font=\scriptsize]{0}}
	[$\frac{4}{38}$, edge label={node[midway,right,font=\scriptsize]{1}}
	[$\frac{2}{38}$, edge label={node[midway,right,font=\scriptsize]{1}}
	[$c_7$, edge label={node[midway,right,font=\scriptsize]{1}}]
	[$c_6$, edge label={node[midway,left,font=\scriptsize]{0}}] ]
	[$c_5$, edge label={node[midway,left,font=\scriptsize]{0}}] ]
	[$c_3$, edge label={node[midway,left,font=\scriptsize]{0}}] ] ]
	[$\frac{23}{38}$, edge label={node[midway,left,font=\scriptsize]{0}} 
	[$c_2$, edge label={node[midway,right,font=\scriptsize]{1}}]
	[$c_1$, edge label={node[midway,left,font=\scriptsize]{0}}] ]
	]
\end{forest}
\end{center}

Der gesamte Bitverbrauch dieser Kodierung ist demnach 32 Bit.

Der durchschnittliche Bitverbrauch f�r ein Symbol betr�gt 
\[\frac{38}{38} + \frac{23}{38} + \frac{15}{38} + \frac{9}{38} + \frac{6}{38} + \frac{4}{38} + \frac{2}{38} + \frac{2}{38} \approx 2.61 \text{ Bit.}\]

Die Entropie betr�gt

\begin{tabular}{lcl}
	$H(X)$ &=& $-(\frac{13}{38} \cdot \log_2(\frac{13}{38}) + \frac{10}{38} \cdot \log_2(\frac{10}{38}) + \frac{5}{38} \cdot \log_2(\frac{5}{38}) + $\\
	&&$\frac{4}{38} \cdot \log_2(\frac{4}{38}) + \frac{2}{38} \cdot \log_2(\frac{2}{38}) + \frac{1}{38} \cdot \log_2(\frac{1}{38}) + $\\
	&&$\frac{1}{38} \cdot \log_2(\frac{1}{38}) + \frac{1}{38} \cdot \log_2(\frac{1}{38}) + \frac{1}{38} \cdot \log_2(\frac{1}{38}))$\\
	&$\approx$& $2.72 \text{ Bit je Symbol}$
\end{tabular}

\begin{tikzpicture}
\begin{axis}[
	title={Bitverbrauch der Huffman-Kodierung},
	xlabel={Anzahl Durchl�ufe},
	ylabel={Anzahl Bits},
	xmin=0, xmax=10000,
	ymin=0, ymax=80,
	xtick={0,20,40,60,80,100},
	ytick={0,10,20,30,40,50,60,70,80},
	legend pos=north west,
	ymajorgrids=true,
	grid style=dashed,
	]
	
	\addplot[
	color=blue,
	]
	coordinates {
		
	};

\end{axis}
\end{tikzpicture}

\section{Grafiken}