%
% Einleitung
%

\chapter{TracePoint Konzept}

Ein neuer Ansatz der speichereffizienten Repr�sentation von Alignments wurde von Gene Myers in \cite{myers} beschrieben und basiert auf dem Konzept der Trace Points.

Sei $A$ ein Alignment von $u[i...j]$ und $v[k...\ell]$ mit $i < j$ und $k < l$ und sei $\Delta \in \mathbb{N}$. Sei $p = \left \lceil\frac{i}{\Delta}\right \rceil$. Man unterteilt $u[i...j]$ in $\tau = \left \lceil\frac{j}{\Delta}\right \rceil - \left \lfloor\frac{i}{\Delta}\right \rfloor$ Substrings $u_0, u_1, ... , u_{\tau -1}$ mit
\[  
u_q =
\begin{cases} 
u[i...p\cdot\Delta] & \text{falls }q = 0 \\
u[(p+q-1)\cdot\Delta+1...(p+q)\cdot\Delta] & \text{falls }0<q<\tau -1\\
u[(p+\tau-2)\cdot\Delta...j] & \text{falls }q = \tau -1
\end{cases}
\]

F�r alle $q$ mit $ 0 \leq q < \tau -1$ sei $t_q$ der letzte Index des Substrings von $v$, der in A mit $u_q$ aligniert. $t_q$ nennt man Trace Point. F�r $q = 0$ aligniert $u_0$ mit $v_0 = v[k...t_0]$. F�r alle $q$ mit $0<q< \tau -1$ aligniert $u_q$ mit $v_q = v[t_{q-1}+1...t_q]$.

Seien $i,j,k,\ell,\Delta$ und die Trace-Points eines Alignments von $u$ und $v$ gegeben. Dann kann ein Alignment $A'$ von $u$ und $v$ mit $\delta (A') \leq \delta (A)$ konstruiert werden. Danach bestimmt man aus den Trace-Points die Substring-Paare $u_q$ und $v_q$, berechnet hierf�r ein optimales Alignment und konkateniert die Alignments von den aufeinanderfolgenden Substring-Paaren zu $A'$.

Beispiel 2:
\begin{verbatim}
Sequenz 1: gagcatgttgcctggtcctttgctaggtactgtagaga
Sequenz 2: gaccaagtaggcgtggaccttgctcggtctgtaagaga
Delta: 15

Gesamtalignment:

0    5    0    5    0    5    0    5    0    
gagc-a-t-gttgcc-tggtcctttgctaggtactgta-gaga
|| | | | |  | | ||| |||| ||| ||| ||||| ||||
gaccaagtag--g-cgtggacctt-gctcggt-ctgtaagaga
0    5    0    5    0    5    0    5    0    



seq1[0...14] aligniert mit seq2[0...15]
gagc-a-t-gttgcc-tgg
|| | | | |  | | |||
gaccaagtag--g-cgtgg

seq1[15...29] aligniert mit seq2[16...28]
tcctttgctaggtac
|||| ||| ||| |
acctt-gctcggt-c

seq1[30...37] aligniert mit seq2[29...37]
tgta-gaga
|||| ||||
tgtaagaga

Trace Points: [15, 28] 

Berechnung der Intervalle anhand der Trace Points:

seq1[0...14] aligniert mit seq2[0...15]
gagc-a-t-gttgcc-tgg
|| | | | |  | | |||
gaccaagtag--g-cgtgg

seq1[15...29] aligniert mit seq2[16...28]
tcctttgctaggtac
|||| ||| ||| |
acctt-gctcggt-c

seq1[30...37] aligniert mit seq2[29...37]
tgta-gaga
|||| ||||
tgtaagaga

0    5    0    5    0    5    0    5    0    
gagc-a-t-gttgcc-tggtcctttgctaggtactgta-gaga
|| | | | |  | | ||| |||| ||| ||| ||||| ||||
gaccaagtag--g-cgtggacctt-gctcggt-ctgtaagaga
0    5    0    5    0    5    0    5    0    
\end{verbatim}

\section{Komplexit�t}

%TODO ausf�hrlicher
 F�r die Trace-Point Repr�sentation wird f�r eine Edit-Distanz $e$ mit Einheitskosten als Kostenfunktion $\delta$ wie oben beschrieben lediglich $O(e^2)$ Zeit pro Teilalignment ben�tigt, wobei bei einer erwarteten Fehlerrate $\varepsilon$ des Alignments die Edit-Distanz immer h�chstens so gro� ist wie die Anzahl der Fehler im Teilalignment. \cite[S.41-42]{gsa-skript}
 
\section{Speicherverbrauch}
\subsection{Beispiel}
Sei $\Delta = 5$ und das Alignment $\cal{A}$
\begin{verbatim}
		0    5    0    5    0    5    0    5    0    
		gagc-a-t-gttgcc-tggtcctttgctaggtactgta-gaga
		|| | | | |  | | ||| |||| ||| ||| ||||| ||||
		gaccaagtag--g-cgtggacctt-gctcggt-ctgtaagaga
		0    5    0    5    0    5    0    5    0   
\end{verbatim}

wie in Abschnitt \ref{Cigar_Speicher} mit den dazugeh�rigen TracePoints $5, 10, 15, 20, 24, 28$ und 34 gegeben. Es ergibt sich somit das Alphabet $\cal{A}$ $= \{5, 10, 15, 20, 24, 28, 34\}$.

Mit einer naiven bin�ren Kodierung br�uchte man demnach f�r jedes Symbol $c \in \cal{A}$ $\left \lceil \log_2(7) \right \rceil = 3$ Bit pro Symbol, also $7 \cdot 3 = 21$ Bit insgesamt.

%TODO un�re Kodierung

Die Repr�sentation des Alignments $\cal{A}$ ben�tigt als CIGAR-String mit einer naiven bin�ren Kodierung 228 Bit und als un�re Kodierung XX Bit, wobei die Kodierung der Trace Points mit $\Delta = 5$ mit einer naiven bin�ren Kodierung 21 Bit und als un�re Kodierung XX Bit ben�tigt.

Es ist somit zu erkennen, dass die Kodierung der Trace Points in diesem Fall mit einem relativ klein gew�hlten $\Delta$ weniger Speicher ben�tigt, als die Kodierung des CIGAR-Strings. Dennoch h�ngt der Speicherverbrauch der Trace Points Kodierung von der Wahl des $\Delta$ ab, da bei einem kleinen $\Delta$ mehr Trace Points und somit Symbole gespeichert werden m�ssen, als bei einem gro�en $\Delta$ und kann somit bei einer sehr ung�nstig gew�hlten Gr��e mehr Speicher verbrauchen als ein CIGAR-String.

\subsection{Delta-Kodierung}

\subsection{Testl�ufe}



\section{Bewertung}
Je gr��er der vorher definierte positive Parameter $\Delta$ ist, desto weniger Trace-Points werden gespeichert und umso l�nger dauert die Berechnung, um die Teil-Alignments zu rekonstruieren. Bei einem kleinen $\Delta$ werden analog mehr Trace-Points gespeichert, aber die Rekonstruktionszeit der Teil-Alignments ist geringer.

Mithilfe von $\Delta$ l�sst sich somit ein Trade-Off zwischen dem Speicherplatzverbrauch und dem Zeitbedarf f�r die Rekonstruktion der Teil-Alignments einstellen.
