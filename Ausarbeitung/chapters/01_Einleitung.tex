%
% Einleitung
%

\chapter{Einleitung}

Ein Sequenzalignment wird in der Bioinformatik dazu verwendet, zwei oder mehrere Sequenzen von zum Beispiel DNA-Str�ngen oder Proteinsequenzen miteinander zu vergleichen und die Verwandtschaft zu bestimmen. Ein Alignment ist das Ergebnis eines solchen Vergleichs. Bei einem globalen Alignment wird jeweils die gesamte Sequenz betrachtet, bei einem lokalen Alignment lediglich Teilabschnitte der beiden Sequenzen.
Um die verschiedenen Sequenzen vergleichen zu k�nnen, berechnet man einen Score oder die Kosten, um den Aufwand, den man betreiben muss, um die gegebenene Sequenz in die Zielsequenz umzuwandeln, beschreiben zu k�nnen. Hierbei wird jeweils das Optimum, also entweder der maximale Score oder die minimalen Kosten gesucht.
Die verschiedenen Schritte, um die Symbole der Strings zu ver�ndern, sind bei Gleichheit ein 'match', bei der Substitution ein 'mismatch', bei der L�schung eine 'deletion' und bei der Einf�gung eine 'insertion', welche je nach Verfahren unterschiedlich gewichtet werden k�nnen. Hierbei haben �hnliche Sequenzen einen hohen Score und geringe Kosten und unterschiedliche Sequenzen analog einen kleinen Score und hohe Kosten.

\section*{Die Edit-Operationen}

Die hier eingef�hrten Begriffe werden in \cite[S. 5-7, 14-16]{gsa-skript} definiert.

Sei $\mathcal{A}$ eine endliche Menge von Buchstaben, die man Alphabet nennt. F�r DNA-Sequenzen verwendet man �blicherweise die Menge der Basen, also $\mathcal{A} = \{a, c, g, t\}$. $\mathcal{A}^i$ sei die Menge der Sequenzen der L�nge $i$ aus $\mathcal{A}$ und $\varepsilon$ sei die leere Sequenz. Formal ausgedr�ckt ist eine Edit-Operation ein Tupel 
\[(\alpha, \beta) \in (\mathcal{A}^1 \cup \{\varepsilon\}) \times (\mathcal{A}^1 \cup \{\varepsilon\}) \backslash \{(\varepsilon, \varepsilon)\},\]

Eine �quivalente Schreibweise von $(\alpha,\beta)$ ist $\alpha \rightarrow \beta$. Es gibt drei verschiedene Edit-Operationen
\begin{align*}
a \rightarrow \varepsilon &\indent \text{ ist eine Deletion f�r alle }a \in \mathcal{A}\\
\varepsilon \rightarrow b &\indent \text{ ist eine Insertion f�r alle }b \in \mathcal{A}\\
a \rightarrow b &\indent \text{ ist eine Substitution f�r alle }a,b \in \mathcal{A}
\end{align*}
Dabei ist zu beachten, dass $\varepsilon \rightarrow \varepsilon$ keine Edit-Operation darstellt.

Ein Alignment von zwei Sequenzen $u$ und $v$ l�sst sich nun als eine Sequenz $(\alpha_1 \rightarrow \beta_1, ... , \alpha_h \rightarrow \beta_h)$ von Edit-Operationen definieren, sodass $u = \alpha_1 ... \alpha_h$ und $v = \beta_1 ... \beta_h$ gilt.

\section*{Die Edit-Distanz}

Sei eine Kostenfunktion $\delta$ mit $\delta(a\rightarrow b)\geq 0$ f�r alle Substitutionen $a \rightarrow b$ und $\delta(\alpha \rightarrow \beta)>0$ f�r alle Einf�gungen und L�schungen $\alpha \rightarrow \beta$ gegeben. Die Kosten f�r ein Alignment $A = (\alpha_1 \rightarrow \beta_1, ... , \alpha_h \rightarrow \beta_h)$ ist die Summe der Kosten aller Edit-Operationen des Alignments.
\[\delta(A) = \sum_{i=1}^{h}\delta(\alpha_i \rightarrow \beta_i)\]
Ein Beispiel einer Kostenfunktion ist die Einheitskostenfunktion
\[
\delta(\alpha \rightarrow \beta) = 
\begin{cases} 
0 ,& \text{wenn } \alpha,\beta \in \mathcal{A} \text{ und } \alpha=\beta \\
1 , & \text{sonst.}
\end{cases}
\]
Die Edit-Distanz von zwei Sequenzen ist wie folgt definiert:
\[edist_\delta(u,v) = \min\{\delta(A) \mid A \text{ ist Alignment von }u \text{ und }v\}\]
Ein Alignment A ist optimal, wenn $\delta(A) = edist_\delta(u,v)$ gilt.\\[0,5cm]
Wenn $\delta$ die Einheitskostenfunktion ist, so ist $edist_\delta(u,v)$ die Levenshtein Distanz \cite[S. 19-21]{gsa-skript}.

Ein Alignment kann f�r eine Edit-Distanz $e$ mit der Einheitskostenfunktion in $O(e)$ Zeit berechnet werden \cite[S. 41-42]{gsa-skript}.